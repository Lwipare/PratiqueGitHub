
\documentclass{sn-jnl}% Default

\usepackage{caption}
\usepackage{natbib}
\bibliographystyle{sn-basic}

\begin{document}

Ici on place un premier bout de texte

	ici c'est la citation  
			\cite{declerck_vitro_2005}et  \cite{fortin_arbuscular_2002}et pour trouver la liste, \cite{silvani_novel_2019}on utilise le 

on est dans la branche numero 3

ici c'est la fin du texte\\

\

\begin{table}[h]
	\caption{Résultat de germination de spores non stériles}\label{res:germ:tabl}%
	\begin{tabular}{0.8\textwidth}{@{}p{2cm}ccrr@{}}
	%	\begin{tabular}{0.8\textwidth}{@{\extracolsep{\hfill}}llllr}
		%\begin{tabular}{lllll}	
		
		
		\toprule
		Experiment/  & Nb   & Nb  & Germination & Total lenght\\
		Specie & spores & germinated & rate &  hyphas mm\\
		\midrule
		Experiment 1 \\
		R. \emph{irregularis}  & 12   & 9  & 75 \% & 16,3 mm  \\
		C. \emph{lamellosum} & 13\footnotemark[1] & 1\footnotemark[1] & 8\% & 40,7 mm  \\
		\hline \\
		Experiment 2 \\
		R. \emph{irregularis}  & 18   & 13  & 72 \% & 16,6 mm  \\
		F. \emph{mossae}   & 18   & 2  & 11 \% & 56,4 mm  \\
		\hline \\
		
		Experiment 3 \\
		R. \emph{irregularis} +\\ 
		M Medium &24 & 16 & 67 \% & 22,9 mm\\
		Acrybase & 24 & 15 & 63 \% & 12,3 mm\\
		H\textsubscript{2}O & 24 & 1 & 4 \% & 2,4 mm\\
		\hline \\
		
		Experiment 4 & Nb of twins\footnotemark[2] & length
		\footnotemark[3]& Sucessrate & \begin{tabular}[x]{@{}c@{}}Foo\\bar\end{tabular} \\
		  & & & & \\
		R. \emph{irregularis}  & 12    & 12  & 100 \% & NA  \\
		F. \emph{fulgida}   & 18   & 2  & 11 \% & NA  \\
		\botrule
	\end{tabular}
	\footnotetext{} \footnotetext[1]{One of the twelve well received two
		linked spores instead of only one. This pair did germinate.}
	\footnotetext[2]{Two not-linked spores were deposited in each
		well.}
	\footnotetext[3]{One or two
		germinated spore gives one germinated well. }
	
\end{table}


%\multicolumn{1}{c}{\multirow{2}{2cm}{Length not measured}}


============%%

\bibliography{zotero}  % Bibliothèque de maitrise de Louis


\end{document}
