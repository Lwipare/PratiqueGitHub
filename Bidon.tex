\documentclass{sn-jnl}% Default
\usepackage{caption}
\usepackage{natbib}
\usepackage{tabularx}
\usepackage{siunitx}
\bibliographystyle{sn-basic}
\begin{document}

Contenu de la version Main pour débuter



	Ici on place un premier bout de texte

 ici c'est la citation \cite{declerck_vitro_2005}et
\cite{fortin_arbuscular_2002}et pour trouver la liste,
\cite{silvani_novel_2019}on utilise le on est dans la branche numero 3 qui est
un très long bout de texte qui se prolonge sur plusieurs lignes et ensuite le
texte est très bizarre ok et pour cela on ne sait pas trop ou ce la va nous
mener car ON DÉPASSE ALLÈGREMEN TLES BOUTS DE LIGNE 



 Les champignons mycorhiziens arbusculaires (CMA) nourrissent les plantes
agricoles, lesquelles nourrissent l'humanité. On connait pourtant encore bien
mal ces champignons car ils sont très difficiles à cultiver et à étudier
\emph{in vitro} en laboratoire avec les outils actuels. L'utilisation d'un
nouveau polymère synthétique transparent et peu couteux dans des protocoles
adaptés ouvre de nouvelles avenues pour les cultiver. Les protocoles proposés
permettent en toute simplicité de cultiver, observer et étudier le champignon
avec/ou sans sa plante hôte sans les contraintes de stérilité des cultures
\textit{in vitro}. Avec ce nouveau type d'outils, plus de laboratoires pourront
les étudier, on connaitra mieux les CMA et \textit{in fin} et on pourra mieux
les utiliser afin de nourrir l'humanité.





enfin j'Mai trouvé du 


	ici c'est la fin du texte\\

	table-

	\begin{table}[h]
		\caption{Composition des solutions}
		\begin{tabular}{@{}lSS@{}}
			\toprule
			& {Milieu M} & {Acrybase}\\
			& {Mg l$ ^{-1} $} &{Mg l$ ^{-1} $} \\
			\midrule
			{MgSO$ _{4} $\textperiodcentered 7H$ _{2} $O}& 731 & 36.55 \\
			{KNO$ _{3} $} & 80 &  \\
			{KH$ _{2} $PO$ _{4} $} & 4.80 & 9.60\\
			{KCl} & 65 & \\
			{(NH$ _{4} $)$ _{2} $SO$ _{4} $} & & 132\\
			{Ca(NO$ _{3} $)$ _{2} $\textperiodcentered 4H$ _{2} $O} & 288 & 28.80 \\
			{NH$ _{4} $NO$ _{3} $} & & 130\\
			\midrule
			{C$ _{10} $H$ _{14} $N$ _{2} $NaO$ _{8} $(E-6760)} & 8 & 8 \\
			KI &0.75 & 0.75 \\
			{MnCl$ _{2} $\textperiodcentered 4(H$ _{2} $O)} & 6.00& 6.00 \\
			{ZnSO$ _{4} $\textperiodcentered7(H$ _{2} $)} & 2.65& 2.65\\
			{H$ _{3} $BO$ _{3} $} & 1.50 & 1.50 \\
			{CuSO$ _{4} $\textperiodcentered7(H$ _{2} $O)} & 0.13& 0.13\\
			{Na$ _{2} $MoO$ _{4} $\textperiodcentered2(H$ _{2} $O)}& 0.0024 & 0.0024 \\
			\bottomrule
		\end{tabular}


	\end{table}



	%\multicolumn{1}{c}{\multirow{2}{2cm}{Length not measured}}


	============%%

	\bibliography{zotero}  % Bibliothèque de maitrise de Louis


\end{document}